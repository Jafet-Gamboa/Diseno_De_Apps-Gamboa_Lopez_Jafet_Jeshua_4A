\documentclass[conference]{IEEEtran}
\IEEEoverridecommandlockouts
% The preceding line is only needed to identify funding in the first footnote. If that is unneeded, please comment it out.
\usepackage{cite}
\usepackage{amsmath,amssymb,amsfonts}
\usepackage{algorithmic}
\usepackage{graphicx}
\usepackage{textcomp}
\usepackage{xcolor}
\def\BibTeX{{\rm B\kern-.05em{\sc i\kern-.025em b}\kern-.08em
    T\kern-.1667em\lower.7ex\hbox{E}\kern-.125emX}}
\begin{document}

\title{Native And Cross-Platform Apps\\
}

\author{\IEEEauthorblockN{ Gamboa López Jafet Jeshua}
\IEEEauthorblockA{\textit{4A} \\
\textit{Universidad Tecnológica de Tijuana}\\
Tijuana, México \\
0322103712@ut-tijuana.edu.mx}
}

\maketitle

\begin{abstract}
This document shows what are the native apps and the cross-platform apps and what are its advantages.
\end{abstract}

\begin{IEEEkeywords}
native, cross-platform, apps, advantages
\end{IEEEkeywords}

\section{Introduction}
Mobile devices have become indispensable companions in our daily lives, and their functionality is intrinsically linked to their hardware components. These physical elements, which form the technological backbone of smartphones and tablets, play a crucial role in the user experience, determining everything from performance to the ability to perform complex tasks.

\section{NATIVE APPS}
A native mobile app is an application that is built for a specific platform. The native mobile application is written in the platform's native programming language: for Android — Kotlin and Java, for Apple iOS — Objective-C and Swift. A native mobile app has access to all the native technologies and hardware capabilities of a particular platform. Native mobile apps must be downloaded and installed on the device, for example through the official Google Play Market and App Store.
\subsection{Advantages}
\begin{itemize} 
\item Access to device hardware (geolocation, camera, microphone, accelerometer, light sensors, calendar, push notifications) and extensive functionality due to this.
\item Can satisfy more different requests from customers and users.
\item User data can be easily collected and analyzed.
\item They generally work more stably and efficiently with any device used on their operating system.
\item There is no limitation on the functionality of the speed and quality of the Internet connection: the application can work without access to the network.
\item Best suited for applications with custom interfaces and complex business logic.
\end{itemize}

\subsection{Disadvantages}
\begin{itemize}
\item Expensive development.
\item Development takes a long time.
\item Every app store must verify native apps.
\item They cover few platforms and are incompatible with other operating systems.
\item Even the smallest changes require regular updates.
\end{itemize}

\section{CROSS-PLATFORM APPS}
Cross-platform app development means that the app is developed using a technology/language/framework that allows it to be used on several different operating systems: Android, iOS, Windows, Linux, etc. For example, React-Native apps can work on Android and iOS.
Hybrid app development means that an app is developed using multiple languages/technologies, but it doesn't always mean that it will be cross-platform. Apps can be hybrid but will not necessarily be considered cross-platform.
An application can be considered cross-platform, but it does not have to be hybrid. It can be a web application or even native (for example, the React Native framework uses a JavaScript runtime to generate JavaScript code and then publish the application to both the Google Play Market and the App Store).
Likewise, apps can be hybrid and cross-platform simultaneously (e.g. React-Native + native platform language).
Approaches in developing a mobile application can be combined. For example, create performance-critical screens on native and secondary technologies on cross-platform ones.

\subsection{Advantages}
\begin{itemize}
\item Cross-platform development is much faster than developing native mobile applications for several different platforms at once.

\item Great for startups that need to get to market faster with an MVP to test a theory.
\item Suitable for creating event applications, for example, for business conferences, trade fairs, etc., due to the speed of creation.

\item Cross-platform development often contributes to more effective development of developers, as it involves working with various technologies and environments and also stimulates problem-solving skills.

\item Cross-platform is useful when writing a simple app with a small number of screens for multiple platforms (a simple mobile game is ideal for cross-platform).
\end{itemize}

\subsection{Disadvantages}
\begin{itemize}
\item iOS and Android differ significantly, and this causes development difficulties and many delays in the work of the finished application (most often this concerns the interface elements and their representation, the Animation FPS and Animation RAM indicators may differ 3 to 5 times).

\item Cross-platform applications crash more frequently and slow down.

\item It is more challenging to maintain cross-platform code: updating systems leads to frequent updating of programming interfaces, which requires more time.

\item In the cross-platform world, there is a small community, and you often must solve problems on your own. There is a high risk of encountering a problem that few people know about.

\item Cross-platform application development can significantly simplify life and save money for a client and business owners who are limited by financial resources and can add headaches to a developer.

\item But a cross-platform application may require great efforts from developers and significant investments from the customer in moving from MVP to a finished product and in scaling the product.

\item A cross-platform application can consume more battery of the user's device, and even one and a half times, which is inconvenient if the application is used frequently.
\end{itemize}





\begin{thebibliography}{00}
\bibitem{b1} R. Walker, “La diferencia entre Las Aplicaciones Móviles Nativas y todas las demás,” AppMaster, 06-May-2022.  [Online]. Available: https://appmaster.io/es/blog/la-diferencia-entre-las-aplicaciones-moviles-nativas-y-todas-las-demas. [Accessed: 15-Jan-2024] 
\end{thebibliography}
\end{document}
